\section{Sudoku}

The Sudoku problem is classically modelled as a constraint problem through the use of \texttt{all\_different} constraints on rows, columns and blocks. Such global inequalities tend to improve upon the use of binary inequalities (see Thibault, M\'emoire de Stage de Master). \texttt{ECLiPSe} and \texttt{CHR} implementations are available in \texttt{code\_file.pl} and in \texttt{code\_file\_chr.pl} respectively. The constraint generating code is fairly trivial and needn't be detailed here. \\\par

Several alternative viewpoints are possible. The widely cited study by Simonis and subsequent studies (that of Laburthe in particular) provide some ideas :\\
 - DUAL\\
 - ABSTRACT\\
 - SAT\\
 - Natural combined model\\
 - Mixed Integer Programming\\
 - Graph colouring problem\\
Laburthe lists rules and models for the puzzles, and proposes a link between the level of difficulty and the constraint models. \\\par

\subsection{Experiments}

Number of backtracks and running time of solvers for each of these models are displayed in table 1. Aside from the provided set of puzzles we also made use of a set of nearly 50.000 puzzles having a minimal amount of pre-filled cells (that would be 17[McGuire, 2014]). \\\par

Demoen shows that in the case of 9x9 Sudokus up to 6 out of the 27 'big' (\texttt{all\_different}) constraints can be redundant[X]. We experimented with the removal of such constraints. This decreased the performance, as expected (footnote to Demoen's study). We didn't experiment further with any of the 'small' constraints.

\subsection{Ideas ...}

 - Probably do ABSTRACT from Laburthe (discuss first), and then some more models "for curiosity's sake". SAT should be useful, fast. These are simple constraints.\\
 - The redundant constraints listed in the study of Helmut Simonis will be useful, and it would be nice to consider his repetitive cycles thing (which he claimed could be a source of more redundant constraints)\\
 - Could make separate LaTeX report with all the code + references to help with navigability\\
 - Simonis also uses inverse constraints to deal with problems with all\_different from ic and ic global, where values are assigned if there's only one left, but not when one value only appears once in all current domains of all variables.