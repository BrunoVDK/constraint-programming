\section{Scheduling Meetings}

The last challenge is the scheduling of some meetings, taking into account the preferences of the various persons involved. A constraint optimization problem where the cost is a function of the end time of the last meeting and the number of `\textit{violations}` (people of lower rank having their meeting after that of people of higher rank). This number of violations is of secondary importance.\\\par

Weekend constraints are generated first. If a person doesn't want to meet on weekends then his or her meeting is not allowed to overlap with the first weekend that follows :
$$((S + \textit{StartingDay})\ \texttt{mod}\ 7) + D <  5$$
In the above constraint $S$ and $D$ represent the start and duration of the person's meeting. Making direct use of \texttt{mod/3} leads to an instantiation error, necessitating the use of an auxiliary variable representing the result of the modulo operation [CODE].\par
Precedence constraints and constraints assuring that no meetings overlap are generated last. The corresponding code is fairly trivial\footnote{Note that all code for this third challenge can be found in \texttt{/src/scheduling/scheduling.pl}.}. The fact that the meeting with the minister should come last is equivalent to adding $N-1$ precedence constraints with $N$ the total number of persons. \\\par

The cost function is defined as $(V_{max}\times E)+V$ where $V_{max}$ is the maximum number of rank violations, $E$ is the end time of the meeting with the minister and $V$ is the actual number of rank violations for a given solution. This ensures that whenever two solutions have a different $E$, the solution with the smallest $E$ will have the lowest cost (whatever the number of violations $V$). Yet if two solutions have the same end time $E$, then it's the number of violations $V$ that will determine what solution is best. Code is displayed in [CODE]. \par
An additional constraint was used for the cost function, stating that it cannot be smaller than $V_{max}\times D_{tot}$ with $D_{tot}$ the sum of all meeting durations. This makes a difference\footnote{In our tests the total runtime for all benchmarks was reduced by a factor of 4.}.\\\par

Some implied constraints were added to increase performance. In case two persons have a different rank but the same meeting duration and weekend preferences, a corresponding order on their start times can safely be imposed. This mustn't override the precedence constraints. The respective implementation is displayed in [CODE].\\\par

Table ??? shows the runtime for each benchmark. Two versions are considered ; one ensures that no two meetings overlap by imposing a \texttt{(}$S_1$+$D_1$ $\leq$ $S_2$\ \texttt{or}\ $S_2$+$D_2$ $\leq$ $S_1$\texttt{)} constraint for every such pair, the other version uses a global version of these same constraints provided by the \texttt{ic\_edge\_finder} library. It's clear that the global version outperforms the other one.\\\par

Instead of making use of implied constraints one can also tinker with the various heuristics provided by the \texttt{search/5} procedure. Some of those lend themselves to some benchmarks but not to others.\par
The \texttt{indomain\_min} heuristic performed better than \texttt{indomain\_max} as it is an optimisation problem after all, meaning that selecting the minimum starting time selects solutions with a smaller cost first. A solution with a lower cost will prune the search tree more.