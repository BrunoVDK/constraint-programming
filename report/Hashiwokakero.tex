\section{Hashiwokakero}

Hashiwokakero is another Japanese logic puzzle published by the same company, in which islands have to be connected by bridges. Six constraints are to be respected, the last one being the connectedness constraint, i.e. that all islands have to be connected. What follows is a discussion of an implementation of two solvers of Hashiwokakero puzzles. One written in ECLiPSe, the other in \texttt{CHR}.

\subsection{ECLiPSe implementation}

A partial solution by Joachim Schimpf was provided. It did not enforce the connectedness constraint. Joachim defines four variables for each of the input puzzle's cells. They represent the number of bridges for each of the cell's directions (north, east, south, west). Then he enforces the five first constraints :
\begin{enumerate}
\item[1-2.] Bridges run in one straight line, horizontally or vertically. This is enforced with equality constraints, making sure that the number of bridges for a given direction of a given cell equals the number of bridges in the opposite direction of a neighbouring cell. A total of four equality constraints for every cell except those on the border, which may only have two or three neighbours\footnote{It can be noted that Joachim's code enforces both $A$\#=$B$ and $B$\#=$A$ in several cases. It has no effect on the runtimes.}.
\item[3.] Bridges cannot cross other bridges or islands. This is enforced by making sure that any cell that does not represent an island either has no horizontal or no vertical bridges.
\item[4.] At most two bridges connect a pair of islands. Joachim imposes this constraint by declaring the domains of the variables to be $[0\dots 2]$.
\item[5.] The number of bridges connected to an island must match the number $X$ on that island. A simple sum constraint ($N+E+S+W$\ \#=\ $X$) suffices to enforce this one.
\end{enumerate}

The connectedness constraint was enforced through the use of an analogous set of four variables ($FN,FE,FS,FW$) per cell, denoting the \textit{flow} for each of the cell's directions. Say the island at the upper left is said to be the sink, then if a flow can be assigned to all islands such that the sink's incoming flow equals the total number of islands minus one, the islands are sure to be connected. The net flow for each non-sink island needs to be one, for each cell it should be zero, and empty cells should have no flow. Most of these constraints can be implemented with equality constraints (the \texttt{ic} library enforces bound consistency for these), some of the others were implemented with the use of the \href{http://eclipseclp.org/doc/bips/lib/ic/EG-2.html}{$\Rightarrow$} (`implication') constraint.\\\par

\begin{snippet}[H]
\caption{Constraint stating that the flow in a bridge should be zero.}\label{hashi}
\small
N + E + S + W \#\textbackslash= 0 => FN \#= -(FS) \texttt{and} FE \#= -(FW) \texttt{and} FN + FE +FS + FW \#= 0
\end{snippet}

All of these constraints are active, meaning that when variables are, in a sense, `woken up', the domain of associated variables is updated accordingly.\\\par

The provided benchmarks are solved rather quickly by the solver. It generally takes a few milliseconds, even for the biggest board. If one makes use of the \texttt{most\_constrained} or the \texttt{occurrence} variable heuristic, runtimes increase. The \texttt{largest} or \texttt{smallest} heuristics perform even worse. This is due to the fact that these heuristics are more likely to label variables. Flow variables have larger domains and most of the values in their domain cannot partake in a solution. As a result, backtracks increase and runtimes do too.

\subsection{CHR implementation}

A \texttt{CHR} solver was also created. Because of the results of the previous experiments no special heuristic (such as \texttt{occurrence} or \texttt{most\_constrained}) was made use of. The solver generates \texttt{island/7} and \texttt{cell/4} constraints which associates islands and cells with their variables (representing the number of bridges in a given direction) and their number (in the case of islands). Every island has a corresponding \texttt{sum/3} constraint which gets updated every time any of the variables in its list is assigned. This corresponds to forward checking. The variables represent the number of bridges for a given direction and a given cell (or island). Instead of defining all these variables separately and enforcing $V_1=V_2$ equality constraints whenever they're supposed to be equal, shared variables are defined instead. This is done in the pre-processing step when the board is read. Because of this decision all but the second and the fifth constraint have to be enforced. The sum constraint was just described. The way the second constraint is dealt with is shown in code snippet \ref{hashi2}. As can be seen, the \texttt{in/2} constraint (and operator) is used to update domains.

\begin{snippet}[H]
\caption{Enforcing that no bridges can cross in \texttt{CHR}.}\label{hashi2}
\small
assign(Val,X), cell(\_,\_,X,Y)\ \#\ passive $\Longrightarrow$ Val > 0\ |\ Y\ in\ [0].\\
assign(Val,X), cell(\_,\_,Y,X)\ \#\ passive $\Longrightarrow$ Val > 0\ |\ Y\ in\ [0].
\end{snippet}

Two additional constraints were added to speed up the solver. The first is a generalised version of the \textit{`4 in the corner, 6 on the side and 8 in the middle'} technique. If an island's number equals twice its number of neighbours then it should be connected with each of these neighbours by a pair of bridges. This covers some of the special cases as \textit{`neighbour'} is defined more broadly as  \textit{`any island to which the island can still be connected with at a certain point'}\footnote{Let's give a concrete example. Say, an island has the number four and three neighbors. Nothing could be concluded before doing any searching. If at any point during search it becomes clear that one of the three neighbors can't be connected to the island, then two pairs of bridges need to be added to form connections with the remaining neighbours.}. As expected, it sped up the solver for all puzzles, almost cutting the total runtime by half.\\\par

The second constraint prevents isolation of some islands by stating that any two neighbouring islands, both with the number one or two, cannot be connected by that same number of bridges\footnote{Note that in the case of trivial boards with nothing else than two such islands enforcing this constraint would prevent the solution from being found.}. Only in the sixth puzzle did this speed up the search by about half a second due to the reduction in number of backtracks. Interestingly the constraint slows down puzzle two, as enforcing it changes the variable ordering leading to a stark increase in number of backtracks\footnote{Experiments with value heuristics (\texttt{indomain\_min}, \texttt{indomain\_max}) showed that changing the heuristic can have a large impact on the number of backtracks. The same holds true for variable heuristics. Part of the reason why ECLiPSe is a whole lot faster is that it enforces bound consistency for the sum constraints. In the sum constraints of the basic \texttt{CHR} solver only forward checking is done.}.\\\par

Some redundant constraints were already part of the basic solver. For example, if an island only has one neighbour, its sum constraint only contains one variable which can readily be assigned.\\\par

As for the connectedness constraints, both passive - and active versions were implemented. Because any initial solution generated without enforcing connectedness still tends to be connected anyways\footnote{At least in the case of the provided benchmarks. In the case of puzzle six and a few smaller boards, the first solution isn't connected.}, any constraint propagation relating to flow tends to slow down the search procedure. Because of this the passive versions of the connectedness constraints outperform the active versions (having a total runtime of about 6 seconds for all six puzzles).\\\par

In the case of puzzle six no solution is found in a reasonable amount of time when the active connectedness constraints are used. The domains of the flow variables are quite large because the number of islands equals 140. While for most of these islands the flow can easily and uniquely be determined, 35 flow sum constraints are less trivial. Generally speaking, when a bunch of islands are connected in a circle (meaning that there is no \textit{`tail'}, i.e. every island is connected to at least 2 other islands) the solution isn't unique. It's what happens in puzzle six, where after cutting all the \textit{`tails'} a few circles remain (see figure \ref{fig:circles}).

\begin{figure}[H]
\centering
\begin{BVerbatim}[fontsize=\footnotesize]
              5 = = 4 = = 4 - - 3 = = 3
              |           |           |
              3 = 3 - 3 = 5           |
                                      3
                                      X
                                      X
          5 - - 2                     3
          X     |                     |
          5 = = 6 - - 6               |
                      X               2
                      X               |
                      6               |
                      X               3
            3 - 4     X               X
            X   X     X               X
            6 = 6 = = 7               4
                      |               X
4 - 4                 |               4
X   |                 |               X
4 = 5                 4 = 3 - 5 = 6 = 4
\end{BVerbatim}
%4 = 5 - 2 - - 3 = 3 - 4 = 3 - 5 = 6 = 4
\caption{Puzzle 6's remnant structure after removing all \textit{`tails'} for which the value of the flow variables is trivial to determine.}
\label{fig:circles}
\end{figure}

If the sum constraints relating to flow would enforce bound consistency (as is the case in ECLiPSe), wrong values could more quickly be filtered out\footnote{A limited form of bound consistency was implemented. A correct solution was generated much more quickly due to that as it helped to realize that a given solution isn't connected. Eventually the search procedure still got stuck on puzzle 6, looking for a valid way to label flow variables. It's harder to ascertain that a given solution \textit{is} connected.}. Doing the same for the other sum constraints (relating to the number of bridges connected to each island) should also reduce the number of backtracks and, therefore, reduce total runtime.